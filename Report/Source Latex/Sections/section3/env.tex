\subsection{Môi trường}
Nhóm đã sử dụng môi trường Google Colaboratory để thực hiện nghiên cứu và phát triển trong bài toán nhận diện người đi bộ. Google Colaboratory (hay gọi tắt là Colab) là một môi trường máy tính trong đám mây miễn phí được cung cấp bởi Google. Nó cho phép người dùng tạo, chia sẻ và thực thi các Jupyter notebook mà không cần cài đặt môi trường trên máy tính cá nhân.

Việc sử dụng Colab giúp nhóm tiết kiệm thời gian và công sức trong việc cài đặt và cấu hình các phần mềm và thư viện phức tạp. Colab cung cấp sẵn nhiều thư viện và công cụ phổ biến cho việc phát triển machine learning và deep learning, bao gồm cả TensorFlow và PyTorch. Bên cạnh đó, nó cũng hỗ trợ việc sử dụng GPU miễn phí để tăng tốc quá trình huấn luyện mô hình.

Với sự linh hoạt và tiện lợi của Colab, nhóm có thể dễ dàng chia sẻ notebook, làm việc đồng thời và tận dụng các tính năng hỗ trợ như lưu trữ dữ liệu trên Google Drive và tích hợp với các dịch vụ cloud khác.

Sử dụng môi trường Google Colaboratory đã giúp nhóm tập trung vào nghiên cứu và thực nghiệm các phương pháp nhận diện người đi bộ mà không bị hạn chế bởi việc cài đặt và cấu hình môi trường phức tạp.